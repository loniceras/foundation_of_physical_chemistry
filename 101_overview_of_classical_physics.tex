\chapter[经典物理概要]{经典物理概要}
%%%%%%%%%%%%%%%%%%%%%%%%%%%%%%%%%%%%%%%%%%%%%%%%%%%%%%%%%%%%%%%%
\section[经典力学]{经典力学}\label{经典力学}
经典力学基于\textbf{参考系}描述力学体系, 参考系通常包括一个坐标系和固定在坐标系中的钟, 其中坐标系可以刻画物体的位置, 钟可以指示时间. \textbf{惯性系}是一种能均匀且各向同性地描述空间, 并能均匀地描述时间的参考系. 一个不受外力的运动物体在惯性系中保持恒定速度, 一个相对某惯性系作匀速直线运动的惯性系也是惯性系. 经典力学遵从\textbf{相对性原理}, 即所有自然定律在所有惯性系中具有相同的形式. 经典非相对论性力学(牛顿力学)使用相互独立的绝对空间和绝对时间来描述力学体系, 且惯性系中物理相互作用的传播速度是无穷大的. 经典相对论性力学使用互相关联的空间和时间来描述力学体系, 且满足\textbf{光速不变原理}, 即惯性系中物理相互作用的最大可能的传播速度是光在真空中的速度$ c $. 对于运动速度远小于光速的物体, 经典非相对论性力学仍然是有效的.

当给定物体的尺寸和形状对运动问题的影响不大时, 可将该物体视作没有尺寸和形状的几何点处理来简化问题, 此即为经典力学中广泛应用的\textbf{质点模型}. 质点在参考坐标系中的位置可用\textbf{径矢}$ \vb*{r} $表示, 质点所处的时间可用$ t $表示. 由多个质点通过某种方式构成的集合称为\textbf{质点系}. 经验表明, 给定质点系中所有质点的位置和速度就可以确定质点系当前的状态, 并且原则上可预测质点系今后的运动. 事实上, 当质点系在运动过程中受到一定\textbf{约束条件}的限制时, 并不需要确定所有质点的位置和速度即可确定质点系当前状态和预测今后运动. 考虑一个具有$ n $个质点的质点系, 对其的约束条件可记作$ f(\vb*{r}_1,\ldots,\vb*{r}_n,\dot{\vb*{r}_1},\dot{\vb*{r}_2},\ldots,\dot{\vb*{r}_n},t)=0 $或$ f(\vb*{r}_1,\ldots,\vb*{r}_n,\dot{\vb*{r}_1},\dot{\vb*{r}_2},\ldots,\dot{\vb*{r}_n},t) \geq 0 $. 前者称为\textbf{双面约束条件}, 后者称为\textbf{单面约束条件}. 双面约束条件可以分为\textbf{完整约束}和\textbf{非完整约束}, 完整约束可减少一个独立坐标和一个独立速度分量, 非完整约束只能减少一个独立速度分量. 完整约束又可分为\textbf{几何约束}和\textbf{可积运动约束}, 几何约束$ f(\vb*{r}_1,\vb*{r}_2,\ldots,\vb*{r}_n,t)=0 $仅对质点系的几何位形加以限制, 可积运动约束同时对质点系的几何位形和运动情况加以限制, 但可以通过对其积分转化为几何约束. 所有约束条件都是完整约束的质点系称为\textbf{完整力学体系}, 否则称为\textbf{非完整力学体系}. 如果约束反力在任意虚位移上所作的功都等于零, 即$ \sum_i \vb*{F}_i \cdot \var{\vb*{r}_i} = 0 $, 则该约束条件称为\textbf{理想约束}. 其中, 约束反力是指约束条件对质点系中各质点产生的反作用力, 虚位移是指质点系中各质点在给定位置满足约束条件的任何无限小位移. 所有约束条件都是理想约束的质点系称为\textbf{理想力学体系}, 否则称为\textbf{非理想力学体系}. 今后我们只考虑理想完整力学体系. 此外, 存在耗散力的体系(即\textbf{非保守体系})已不是纯粹的力学体系, 因其复杂的微观机制而通常不能被直接处理, 但对一些简单的情形可近似地用力学方法处理.

给定具有$ n $个质点和$ k $个相互独立的完整约束的理想完整力学体系, 我们至少需要$ 3n-k $个参数才能确定某时刻该力学体系中各质点位置, 即确定力学体系的位形. 我们称能够确定位形的最少参数$ 3n-k $为该力学体系的\textbf{独立广义坐标数}, 称能够唯一确定位形的一组$ 3n-k $个参数为该力学体系的\textbf{广义坐标}$ q $. 应当注意, 力学体系的广义坐标仅代表一些互相独立的参数, 它们可以是距离或角度等常见物理量, 也可以根本没有几何意义, 完整力学体系的\textbf{自由度}即为独立广义坐标数. 广义坐标$ q $对时间$ t $的导数$ \dv*{q}{t} $称为\textbf{广义速度}, 记作$ \dot{q} $. 广义坐标$ q $对时间$ t $的二阶导数$ \dv*[2]{q}{t} $称为\textbf{广义加速度}, 记作$ \ddot{q} $. 如果一个独立广义坐标数为$ s $的力学体系的广义坐标为$ (q_1,q_2,\ldots,q_s) $, 则其广义速度为$ (\dot{q_1},\dot{q_2},\ldots,\dot{q_s}) $, 广义加速度为$ (\ddot{q_1},\ddot{q_2},\ldots,\ddot{q_s}) $, 它们可形式上分别记作列向量$ \vb*{q},\dot{\vb*{q}},\ddot{\vb*{q}} $. 仿照质点的径矢的概念, 我们也可以对力学体系定义\textbf{广义位矢}$ (\vb*{r}_1(\vb*{q},t),\vb*{r}_2(\vb*{q},t),\ldots,\vb*{r}_n(\vb*{q},t)) $.

\begin{postulate}[最小作用量原理]\label{pos:最小作用量原理}
    每个力学体系可以由一个确定的函数$ \lagr(\vb*{q},\dot{\vb*{q}},t) $表征, 其中$ \vb*{q} $是广义坐标, $ \dot{\vb*{q}} $是广义速度, $ t $是时间. 假设力学体系在$ t=t_{\init} $时广义坐标$ \vb*{q}_{\init} $和$ t=t_{\fini} $时的广义坐标$ \vb*{q}_{\fini} $是给定的, 则该体系由运动方程确定的真实运动轨迹$ \vb*{q}(t) $一定使得
    \begin{equation}
        \action(\vb*{q}_{\init},t_{\init};\vb*{q}_{\fini},t_{\fini})=\int_{t_{\init}}^{t_{\fini}} \lagr(\vb*{q},\dot{\vb*{q}},t) \dd{t}
    \end{equation}
    取极值(通常是极小值, 但也可能是鞍点或极大值).
\end{postulate}
其中, $ \lagr(\vb*{q},\dot{\vb*{q}},t) $称为体系的\textbf{拉格朗日函数}, $ \action(\vb*{q}_{\init},t_{\init};\vb*{q}_{\fini},t_{\fini}) $称为体系的\textbf{哈密顿作用量}.

我们运用变分法来获得体系在拉格朗日力学下的运动方程. 考虑哈密顿作用量$ \action(\vb*{q}_{\init},t_{\init};\vb*{q}_{\fini},t_{\fini}) $的等时变分
\begin{align*}
    \var{\action(\vb*{q}_{\init},t_{\init};\vb*{q}_{\fini},t_{\fini})} &=\int_{t_{\init}}^{t_{\fini}} \lagr(\vb*{q},\dot{\vb*{q}},t) \dd{t} = \int_{t_{\init}}^{t_{\fini}} \sum_{\beta=1}^s\bigg(\pdv{\lagr}{q_{\beta}}\var{q_{\beta}}+\pdv{\lagr}{\dot{q_{\beta}}}\var{\dot{q_{\beta}}}\bigg) \dd{t}\\
    &=\int_{t_{\init}}^{t_{\fini}} \sum_{\beta=1}^s\bigg(\pdv{\lagr}{q_{\beta}}\var{q_{\beta}} + \dv{t}(\pdv{\lagr}{\dot{q_{\beta}}} \var{q_{\beta}}) - \var{q_{\beta}} \dv{t}(\pdv{\lagr}{\dot{q_{\beta}}})\bigg) \dd{t}\\
    &=\eval{\sum_{\beta=1}^s \pdv{\lagr}{\dot{q_{\beta}}}\var{q_{\beta}}}_{t_{\init}}^{t_{\fini}}+\int_{t_{\init}}^{t_{\fini}} \bigg(\pdv{\lagr}{q_{\beta}} - \dv{t}(\pdv{\lagr}{\dot{q_{\beta}}})\bigg) \var{q_{\beta}} \dd{t}.
\end{align*}
注意, 对等时变分有$ \var{\dot{q_{\alpha}}} = \dv{t} \var{q_{\alpha}} $. 由于$ \var{\vb*{q}(t_{\init})}=\var{\vb*{q}(t_{\fini})}=0 $, 得$ \eval{\sum_{\beta=1}^s \pdv{\lagr}{\dot{q_{\beta}}}\var{q_{\beta}}}_{t_{\init}}^{t_{\fini}}=0 $. 为了使$ \action(\vb*{q}_{\init},t_{\init};\vb*{q}_{\fini},t_{\fini}) $取极小值, $ \var{\action(\vb*{q}_{\init},t_{\init};\vb*{q}_{\fini},t_{\fini})}=0 $. 由于$ \var{q_{\beta}} $的任意性, 得对任意$ \alpha \in \{1,2,\ldots,s\} $有\textbf{欧拉}--\textbf{拉格朗日方程}:
\begin{equation}
    \pdv{\lagr}{q_{\alpha}} - \dv{t}(\pdv{\lagr}{\dot{q_{\alpha}}}) = 0.
\end{equation}
力学体系的拉格朗日函数$ \lagr(\vb*{q},\dot{\vb*{q}},t) $具有\textbf{规范不定性}, 即在\textbf{定域规范变换}$ \tilde{\lagr}(\vb*{q},\dot{\vb*{q}},t)=\lagr(\vb*{q},\dot{\vb*{q}},t)+\dv*{f(\vb*{q},t)}{t} $下有$ \var{\tilde{\action}(\vb*{q}_{\init},t_{\init};\vb*{q}_{\fini},t_{\fini})}=\var{\action(\vb*{q}_{\init},t_{\init};\vb*{q}_{\fini},t_{\fini})} $, 进而在定域规范变换下体系的欧拉{--}拉格朗日方程和运动规律也不变, 此即为体系的\textbf{规范不变性}.

通过拉格朗日变量定义\textbf{广义动量}$ p=\pdv*{\lagr}{\dot{q}} $, 得\textbf{广义动量定理}: 对任意$ \alpha \in \{1,2,\ldots,s\} $有$ \dot{p_{\alpha}}=\pdv*{\lagr}{q_{\alpha}} $. 在定域规范变换$ \tilde{\lagr}(\vb*{q},\dot{\vb*{q}},t)=\lagr(\vb*{q},\dot{\vb*{q}},t)+\dv*{f(\vb*{q},t)}{t} $下, 广义动量为$ \tilde{p_{\alpha}}=p_{\alpha}+\pdv*{f}{q_{\alpha}} $, 因此广义动量也具有规范不定性, 其值取决于拉格朗日函数的规范选择. 考虑力学体系的所有质点在空间内整体平移相同的位移, 不妨对任意$ i \in \{1,2,\ldots,n\} $都有$ \pdv*{\vb*{r}_i}{q_{\alpha}}=\vb*{e} $且$ \var{\vb*{r}_i} = \sum_{\beta=1}^s \pdv{\vb*{r}_i}{q_{\beta}} \var{q_{\beta}} = \var{q_{\alpha}} \vb*{e} $. 如果该体系的所有质点在空间内整体平移后的拉格朗日函数仍保持不变(即体系满足\textbf{空间平移不变性}或\textbf{空间均匀性}), 即$ \pdv*{\lagr}{q_{\alpha}}=0 $, 则$ \dot{p_{\alpha}}=0 $. 考虑力学体系的所有质点在空间内整体旋转相同的角度, 不妨对任意$ i \in \{1,2,\ldots,n\} $都有$ \pdv*{\vb*{r}_i}{q_{\alpha}}=\vb*{e} \times \vb*{r}_i $且$ \var{\vb*{r}_i} = \sum_{\beta=1}^s \pdv{\vb*{r}_i}{q_{\beta}} \var{q_{\beta}} = \var{q_{\alpha}} (\vb*{e} \times \vb*{r}_i) $. 如果该体系的所有质点在空间内整体旋转后的拉格朗日函数仍保持不变(即体系满足\textbf{空间转动不变性}或\textbf{空间各向同性}), 即$ \pdv*{\lagr}{q_{\alpha}}=0 $, 则$ \dot{p_{\alpha}}=0 $. 现在我们考虑$ \lagr(\vb*{q},\dot{\vb*{q}},t) $对时间的导数
\begin{align*}
    \dv{\lagr}{t} &= \sum_{\beta=1}^s \bigg(\pdv{\lagr}{q_{\beta}}\dot{q_{\beta}} + \pdv{\lagr}{\dot{q_{\beta}}} \ddot{q_{\beta}} \bigg) + \pdv{\lagr}{t} = \sum_{\beta=1}^s \pdv{\lagr}{q_{\beta}}\dot{q_{\beta}} + \dv{t} \sum_{\beta=1}^s \pdv{\lagr}{\dot{q_{\beta}}} \dot{q_{\beta}} - \sum_{\beta=1}^s \dv{t}(\pdv{\lagr}{\dot{q_{\beta}}}) \dot{q_{\beta}} + \pdv{\lagr}{t} \\
    &=\sum_{\beta=1}^s \bigg(\pdv{\lagr}{q_{\beta}} - \dv{t}(\pdv{\lagr}{\dot{q_{\beta}}}) \bigg)\dot{q_{\beta}} + \dv{t} \sum_{\beta=1}^s \pdv{\lagr}{\dot{q_{\beta}}} \dot{q_{\beta}} + \pdv{\lagr}{t} = \dv{t} \sum_{\beta=1}^s p_{\beta} \dot{q_{\beta}} + \pdv{\lagr}{t}.
\end{align*}
通过拉格朗日变量定义\textbf{广义能量}$ E=\sum_{\beta=1}^s p_{\beta}\dot{q_{\beta}} - \lagr(\vb*{q},\dot{\vb*{q}},t) $, 得\textbf{广义能量定理}: $ \dot{E}=-\pdv*{\lagr}{t} $. 在定域规范变换$ \tilde{\lagr}(\vb*{q},\dot{\vb*{q}},t)=\lagr(\vb*{q},\dot{\vb*{q}},t)+\dv*{f(\vb*{q},t)}{t} $下, 广义能量为$ \tilde{E}=E-\pdv*{f}{t} $, 因此广义能量也具有规范不定性, 其值取决于拉格朗日函数的规范选择. 考虑力学体系的所有质点在时间内整体平移, 如果该体系时间在平移后的拉格朗日函数仍保持不变(即体系满足\textbf{时间平移不变性}或\textbf{时间均匀性}), 即$ \pdv*{\lagr}{t}=0 $, 则$ \dot{E}=0 $.

类似地, 我们运用变分法来获得体系在哈密顿力学下的运动方程. 定义\textbf{哈密顿函数}$ \hami(\vb*{q},\vb*{p},t) $满足
\begin{equation}
    \hami(\vb*{q},\vb*{p},t) + \lagr(\vb*{q},\dot{\vb*{q}},t) = \sum_{\beta=1}^s p_{\beta}\dot{q_{\beta}} = \vb*{p} \cdot \dot{\vb*{q}}.
\end{equation}
由于广义动量具有规范不定性, 其值取决于拉格朗日函数的规范选择, 因此我们可将广义动量视作与广义坐标地位平等的独立变量. 考虑哈密顿作用量$ \action(\vb*{q}_{\init},\vb*{p}_{\init},t_{\init};\vb*{q}_{\fini},\vb*{p}_{\fini},t_{\fini}) $的等时变分
\begin{align*}
    \var{\action(\vb*{q}_{\init},\vb*{p}_{\init},t_{\init};\vb*{q}_{\fini},\vb*{p}_{\fini},t_{\fini})} &=\int_{t_{\init}}^{t_{\fini}} \lagr(\vb*{q},\dot{\vb*{q}},t) \dd{t} = \int_{t_{\init}}^{t_{\fini}} \sum_{\beta=1}^s\bigg(p_{\beta} \var{\dot{q_{\beta}}} + \dot{q_{\beta}} \var{p_{\beta}} - \pdv{\hami}{q_{\beta}}\var{q_{\beta}} - \pdv{\hami}{p_{\beta}}\var{p_{\beta}}\bigg) \dd{t}\\
    &=\int_{t_{\init}}^{t_{\fini}} \bigg(\dv{t} \sum_{\beta=1}^s p_{\beta} \var{q_{\beta}} +\sum_{\beta=1}^s \bigg(\dot{q_{\beta}} \var{p_{\beta}} - \dot{p_{\beta}} \var{q_{\beta}} - \pdv{\hami}{q_{\beta}}\var{q_{\beta}} - \pdv{\hami}{p_{\beta}}\var{p_{\beta}} \bigg)\bigg) \dd{t}\\
    &=\eval{\sum_{\beta=1}^s p_{\beta} \var{q_{\beta}}}_{t_{\init}}^{t_{\fini}} + \int_{t_{\init}}^{t_{\fini}} \bigg(\sum_{\beta=1}^s \bigg(\dot{q_{\beta}} - \pdv{\hami}{p_{\beta}} \bigg)\var{p_{\beta}} - \sum_{\beta=1}^s \bigg(\dot{p_{\beta}} + \pdv{\hami}{q_{\beta}} \bigg)\var{q_{\beta}} \bigg) \dd{t}.
\end{align*}
注意, 对等时变分有$ \var{\dot{q_{\alpha}}} = \dv{t} \var{q_{\alpha}} $. 由于$ \var{\vb*{q}(t_{\init})}=\var{\vb*{q}(t_{\fini})}=0 $, 得$ \eval{\sum_{\beta=1}^s p_{\beta} \var{q_{\beta}}}_{t_{\init}}^{t_{\fini}}=0 $. 为了使$ \action(\vb*{q}_{\init},\vb*{p}_{\init},t_{\init};\vb*{q}_{\fini},\vb*{p}_{\fini},t_{\fini}) $取极小值, $ \var{\action(\vb*{q}_{\init},\vb*{p}_{\init},t_{\init};\vb*{q}_{\fini},\vb*{p}_{\fini},t_{\fini})}=0 $. 由于$ \var{q_{\beta}} $和$ \var{p_{\beta}} $的任意性, 得对任意$ \alpha \in \{1,2,\ldots,s\} $有\textbf{哈密顿方程}:
\begin{equation}
    \pdv{\hami}{p_{\alpha}} = \dot{q_{\alpha}}, \quad \pdv{\hami}{q_{\alpha}} = -\dot{p_{\alpha}}.
\end{equation}
拉格朗日函数$ \lagr(\vb*{q},\dot{\vb*{q}},t) $的规范不定性也可以推广至在定域规范变换$ \tilde{\lagr}(\vb*{q},\dot{\vb*{q}},t)=\lagr(\vb*{q},\dot{\vb*{q}},t)+\dv*{f(\vb*{q},\vb*{p},t)}{t} $下有$ \var{\action(\vb*{q}_{\init},\vb*{p}_{\init},t_{\init};\vb*{q}_{\fini},\vb*{p}_{\fini},t_{\fini})}=\var{\action(\vb*{q}_{\init},\vb*{p}_{\init},t_{\init};\vb*{q}_{\fini},\vb*{p}_{\fini},t_{\fini})} $. 对拉格朗日函数$ \lagr(\vb*{q},\dot{\vb*{q}},t) $进行勒让德变换, 也可以获得哈密顿函数$ \hami(\vb*{q},\vb*{p},t) $和哈密顿方程, 即由
\begin{equation*}
    \dd{\hami} = \sum_{\beta=1}^s (p_{\beta} \dd{\dot{q_{\beta}}} + \dot{q_{\beta}} \dd{p_{\beta}}) - \sum_{\beta=1}^s \bigg(\dot{p_{\beta}} \dd{q_{\beta}} + p_{\beta} \dd{\dot{q_{\beta}}} + \pdv{\lagr}{t}\dd{t}\bigg) = \sum_{\beta=1}^s \bigg(\dot{q_{\beta}} \dd{p_{\beta}} - \dot{p_{\beta}} \dd{q_{\beta}} - \pdv{\lagr}{t}\dd{t}\bigg)
\end{equation*}
知对任意$ \alpha \in \{1,2,\ldots,s\} $有$ \pdv*{\hami}{p_{\alpha}} = \dot{q_{\alpha}} $和$ \pdv*{\hami}{q_{\alpha}} = -\dot{p_{\alpha}} $. 由$ \pdv*{\hami}{q_{\alpha}} = -\pdv*{\lagr}{q_{\alpha}} $可知, 体系的拉格朗日函数不显含某个广义坐标当且仅当其哈密顿函数也不显含该广义坐标. 此外, 易发现$ \pdv*{\hami}{t} = -\pdv*{\lagr}{t} $, 因而体系的拉格朗日函数不显含时间当且仅当体系的哈密顿函数不显含时间. 考虑$ \hami(\vb*{q},\vb*{p},t) $对时间的导数, 有
\begin{equation*}
    \dv{\hami}{t} = \sum_{\beta =1}^s \bigg(\pdv{\hami}{q_{\beta}}\dot{q_{\beta}} + \pdv{\hami}{p_{\beta}}\dot{p_{\beta}}\bigg) + \pdv{\hami}{t} = \pdv{\hami}{t}.
\end{equation*}
易发现哈密顿函数$ \hami(\vb*{q},\vb*{p},t) $即为采用哈密顿变量表示的广义能量. 如果体系的哈密顿函数$ \hami(\vb*{q},\vb*{p},t) $不显含时间, 则该体系在运动过程中恒有广义能量$ \hami(\vb*{q},\vb*{p},t) $为定值$ E $, 并称该类体系是\textbf{广义保守的}.

对拉格朗日函数$ \lagr(\vb*{\zeta},\vb*{q},\dot{\vb*{\zeta}},\dot{\vb*{q}},t)=\lagr(\zeta_1,\zeta_2,\ldots,\zeta_r,q_{r+1},q_{r+2},\ldots,q_s,\dot{\zeta_1},\dot{\zeta_2},\ldots,\dot{\zeta_r},\dot{q_{r+1}},\dot{q_{r+2}},\ldots,\dot{q_s},t) $进行勒让德变换时, 我们可以仅将部分拉格朗日变量$ \vb*{q},\dot{\vb*{q}} $转化为哈密顿变量$ \vb*{q},\vb*{p} $, 最终得到具有组合变量的罗斯函数$ \routh(\vb*{\zeta},\vb*{q},\dot{\vb*{\zeta}},\vb*{p},t)=\routh(\zeta_1,\zeta_2,\ldots,\zeta_r,q_{r+1},q_{r+2},\ldots,q_s,\dot{\zeta_1},\dot{\zeta_2},\ldots,\dot{\zeta_r},p_{r+1},p_{r+2},\ldots,p_s,t) $和罗斯方程. 定义\textbf{罗斯函数}满足
\begin{equation}
    \routh(\vb*{\zeta},\vb*{q},\dot{\vb*{\zeta}},\vb*{p},t) + \lagr(\vb*{\zeta},\vb*{q},\dot{\vb*{\zeta}},\dot{\vb*{q}},t) = \sum_{\beta=r+1}^s p_{\beta}\dot{q_{\beta}} = \vb*{p} \cdot \dot{\vb*{q}},
\end{equation}
现对$ \routh(\vb*{\zeta},\vb*{q},\dot{\vb*{\zeta}},\vb*{p},t) $作全微分, 得
\begin{align*}
    \dd{\routh} &= \sum_{\beta=r+1}^s (p_{\beta} \dd{\dot{q_{\beta}}} + \dot{q_{\beta}} \dd{p_{\beta}}) - \sum_{\beta=1}^r \bigg(\pdv{\lagr}{\xi_{\beta}}\dd{\xi_{\beta}} + \pdv{\lagr}{\dot{\xi_{\beta}}}\dd{\dot{\xi_{\beta}}}\bigg) - \sum_{\beta=r+1}^s \bigg(\pdv{\lagr}{q_{\beta}}\dd{q_{\beta}} + \pdv{\lagr}{\dot{q_{\beta}}}\dd{\dot{q_{\beta}}}\bigg) - \pdv{\lagr}{t}\dd{t} \\
    &= \sum_{\beta=1}^r \bigg(- \pdv{\lagr}{\xi_{\beta}}\dd{\xi_{\beta}} - \pdv{\lagr}{\dot{\xi_{\beta}}}\dd{\dot{\xi_{\beta}}}\bigg) + \sum_{\beta=r+1}^s (\dot{q_{\beta}}\dd{p_{\beta}} - \dot{p_{\beta}}\dd{q_{\beta}}) - \pdv{\lagr}{t}\dd{t}.
\end{align*}
因此, 对任意$ \alpha \in \{1,2,\ldots,r\} $有\textbf{罗斯方程}:
\begin{equation}
    \pdv{\routh}{\xi_{\alpha}} = -\pdv{\lagr}{\xi_{\alpha}}, \quad \pdv{\routh}{\dot{\xi_{\alpha}}} = -\pdv{\lagr}{\dot{\xi_{\alpha}}}, \quad\pdv{\routh}{\xi_{\alpha}} - \dv{t}(\pdv{\routh}{\dot{\xi_{\alpha}}}) = 0;
\end{equation}
对任意$ \alpha \in \{r+1,r+2,\ldots,s\} $有\textbf{罗斯方程}:
\begin{equation}
    \pdv{\routh}{p_{\alpha}} = \dot{q_{\alpha}}, \quad \pdv{\routh}{q_{\alpha}} = -\dot{p_{\alpha}}.
\end{equation}
由$ \pdv*{\routh}{\xi_{\alpha}} = -\pdv*{\lagr}{\xi_{\alpha}} $和$ \pdv*{\routh}{q_{\alpha}} = -\pdv*{\lagr}{q_{\alpha}} $可知, 体系的拉格朗日函数不显含某个广义坐标当且仅当其罗斯函数也不显含该广义坐标. 此外, 易发现$ \pdv*{\routh}{t} = -\pdv*{\lagr}{t} $, 因而体系的拉格朗日函数不显含时间当且仅当体系的罗斯函数不显含时间. 考虑$ \routh(\vb*{\zeta},\vb*{q},\dot{\vb*{\zeta}},\vb*{p},t) $对时间的导数, 有
\begin{align*}
    \dv{\routh}{t} &= \dv{t} \sum_{\beta=r+1}^s p_{\beta}\dot{q_{\beta}} - \bigg(\dv{t} \sum_{\beta=1}^r \pdv{\lagr}{\xi_{\beta}} \dot{\xi_{\beta}} + \dv{t} \sum_{\beta=r+1}^s p_{\beta} \dot{q_{\beta}} + \pdv{\lagr}{t}\bigg) \\
    &= -\dv{t} \sum_{\beta=1}^r \pdv{\lagr}{\dot{\xi_{\beta}}} \dot{\xi_{\beta}} - \pdv{\lagr}{t} = \dv{t} \sum_{\beta=1}^r \pdv{\routh}{\dot{\xi_{\beta}}} \dot{\xi_{\beta}} + \pdv{\routh}{t},
\end{align*}
得$ \dv{t} \big(\routh - \sum_{\beta=1}^r \pdv{\routh}{\dot{\xi_{\beta}}}\dot{\xi_{\beta}}\big) = \pdv*{\routh}{t} $. 易发现$ \routh(\vb*{\zeta},\vb*{q},\dot{\vb*{\zeta}},\vb*{p},t) - \sum_{\beta=1}^r \pdv{\routh}{\dot{\xi_{\beta}}}\dot{\xi_{\beta}} $即为采用罗斯变量表示的广义能量. 如果体系的罗斯函数$ \routh(\vb*{\zeta},\vb*{q},\dot{\vb*{\zeta}},\vb*{p},t) $不显含时间, 则该体系在运动过程中恒有广义能量$ \routh(\vb*{\zeta},\vb*{q},\dot{\vb*{\zeta}},\vb*{p},t) - \sum_{\beta=1}^r \pdv{\routh}{\dot{\xi_{\beta}}}\dot{\xi_{\beta}} $为定值$ E $.

我们现在考虑拉格朗日函数和哈密顿函数的独立变量变换问题. 考虑对拉格朗日函数$ \lagr(\vb*{q},\dot{\vb*{q}},t) $的广义坐标$ \vb*{q} $作\textbf{点变换}满足$ \vb*{q}=\vb*{q}(\vb*{Q},t) $, 此时自然有$ \lagr_{\vb*{Q}}(\vb*{Q},\dot{\vb*{Q}},t)=\lagr(\vb*{q}(\vb*{Q},t),\dot{\vb*{q}}(\vb*{Q},\dot{\vb*{Q}},t),t) $. 可以发现欧拉{--}拉格朗日方程在点变换下保持不变, 即对任意$ \alpha \in \{1,2,\ldots,s\} $有
\begin{align*}
    \pdv{\lagr_{\vb*{Q}}}{Q_{\alpha}} - \dv{t}(\pdv{\lagr_{\vb*{Q}}}{\dot{Q_{\alpha}}}) &= \sum_{\beta=1}^s \pdv{\lagr}{q_{\beta}}\pdv{q_{\beta}}{Q_{\alpha}} + \sum_{\beta=1}^s \pdv{\lagr}{\dot{q_{\beta}}}\pdv{\dot{q_{\beta}}}{Q_{\alpha}} - \dv{t} \sum_{\beta=1}^s \pdv{\lagr}{\dot{q_{\beta}}}\pdv{\dot{q_{\beta}}}{\dot{Q_{\alpha}}}\\
    &=\sum_{\beta=1}^s \bigg(\pdv{\lagr}{q_{\beta}}\pdv{q_{\beta}}{Q_{\alpha}} - \pdv{\dot{q_{\beta}}}{\dot{Q_{\alpha}}}\dv{t}\pdv{\lagr}{\dot{q_{\beta}}} \bigg) + \sum_{\beta=1}^s \pdv{\lagr}{\dot{q_{\beta}}} \bigg(\pdv{\dot{q_{\beta}}}{Q_{\alpha}} - \dv{t} \pdv{\dot{q_{\beta}}}{\dot{Q_{\alpha}}} \bigg)\\
    &=\sum_{\beta=1}^s \bigg(\pdv{\lagr}{q_{\beta}} - \dv{t}\pdv{\lagr}{\dot{q_{\beta}}} \bigg) \pdv{q_{\beta}}{Q_{\alpha}} + \sum_{\beta=1}^s \pdv{\lagr}{\dot{q_{\beta}}} \bigg(\pdv{\dot{q_{\beta}}}{Q_{\alpha}} - \dv{t} \pdv{q_{\beta}}{Q_{\alpha}} \bigg)=0.
\end{align*}
因此, 哈密顿方程在点变换下也保持不变, 即对任意$ \alpha \in \{1,2,\ldots,s\} $都有$ \pdv*{\hami_{\vb*{Q}}}{P_{\alpha}} = \dot{Q_{\alpha}} $和$ \pdv*{\hami_{\vb*{Q}}}{Q_{\alpha}} = -\dot{P_{\alpha}} $, 其中$ P_{\alpha}=\pdv*{\lagr_{\vb*{Q}}}{\dot{Q_{\alpha}}} $. 罗斯方程在点变换下也有类似的结论. 由于哈密顿函数的广义坐标和广义动量都是独立的变量, 我们实际上可以考虑更广泛的变换. 对哈密顿函数$ \hami(\vb*{q},\vb*{p},t) $的广义坐标$ \vb*{q} $和广义动量$ \vb*{p} $作变换满足$ \vb*{q}=\vb*{q}(\vb*{Q},\vb*{P},t) $和$ \vb*{p}=\vb*{p}(\vb*{Q},\vb*{P},t) $, 此时自然有$ \hami_{\vb*{QP}}(\vb*{Q},\vb*{P},t)=\hami(\vb*{q}(\vb*{Q},\vb*{P},t),\vb*{p}(\vb*{Q},\vb*{P},t),t) $. 如果哈密顿方程能在该变换下保持不变, 即对任意$ \alpha \in \{1,2,\ldots,s\} $都有$ \pdv*{\hami_{\vb*{QP}}}{P_{\alpha}} = \dot{Q_{\alpha}} $和$ \pdv*{\hami_{\vb*{QP}}}{Q_{\alpha}} = -\dot{P_{\alpha}} $, 则称该变换为\textbf{正则变换}. 通过正则变换产生的独立变量$ \vb*{Q},\vb*{P} $显然应当满足最小作用量原理, 应当考虑
\begin{align*}
    \var{\action(\vb*{q}_{\init},\vb*{p}_{\init},t_{\init};\vb*{q}_{\fini},\vb*{p}_{\fini},t_{\fini})} &= \var{\int_{t_{\init}}^{t_{\fini}} \bigg(\sum_{\beta=1}^s p_{\beta}\dot{q_{\beta}} - \hami(\vb*{q},\vb*{p},t) + \dv{t} f_1(\vb*{q},\vb*{p},t)\bigg) \dd{t}} = 0,\\
    \var{\action(\vb*{Q}_{\init},\vb*{P}_{\init},t_{\init};\vb*{Q}_{\fini},\vb*{P}_{\fini},t_{\fini})} &= \var{\int_{t_{\init}}^{t_{\fini}} \bigg(\sum_{\beta=1}^s P_{\beta}\dot{Q_{\beta}} - \hami_{\vb*{Q}\vb*{P}}(\vb*{Q},\vb*{P},t) + \dv{t} f_2(\vb*{Q},\vb*{P},t)\bigg) \dd{t}} = 0,
\end{align*}
以上两式相减时引入标度因子$ \lambda $, 定义第一类母函数$ F_1=F_1(\vb*{q},\vb*{Q},t)=f_2(\vb*{Q},\vb*{P},t)-\lambda f_1(\vb*{q},\vb*{p},t) $, 得
\begin{equation*}
    \var{\int_{t_{\init}}^{t_{\fini}} \bigg(\lambda \bigg(\sum_{\beta=1}^s p_{\beta}\dot{q_{\beta}} - \hami(\vb*{q},\vb*{p},t)\bigg) - \bigg(\sum_{\beta=1}^s P_{\beta}\dot{Q_{\beta}} - \hami_{\vb*{Q}\vb*{P}}(\vb*{Q},\vb*{P},t)\bigg) - \dv{F_1}{t} \bigg) \dd{t}}=0.
\end{equation*}
要使上式成立, 只要
\begin{equation*}
    \lambda \bigg(\sum_{\beta=1}^s p_{\beta}\dot{q_{\beta}} - \hami(\vb*{q},\vb*{p},t)\bigg) = \sum_{\beta=1}^s P_{\beta}\dot{Q_{\beta}} + \hami_{\vb*{Q}\vb*{P}}(\vb*{Q},\vb*{P},t) + \dot{F_1} = 0,
\end{equation*}
此时有
\begin{equation*}
    \dd{F_1}(\vb*{q},\vb*{Q},t) = \lambda \sum_{\beta=1}^s p_{\beta} \dd{q_{\beta}} - \sum_{\beta=1}^s P_{\beta} \dd{Q_{\beta}} + \big(\hami_{\vb*{Q}\vb*{P}}(\vb*{Q},\vb*{P},t) - \lambda \hami(\vb*{q},\vb*{p},t)\big) \dd{t}.
\end{equation*}
对$ F_1(\vb*{q},\vb*{Q},t) $作勒让德变换可获得另外三类母函数. 定义第二类母函数$ F_2=F_2(\vb*{q},\vb*{P},t)=F_1+\sum_{\beta=1}^s P_{\beta}Q_{\beta} $, 第三类母函数$ F_3=F_3(\vb*{p},\vb*{Q},t)=F_1-\lambda\sum_{\beta=1}^s p_{\beta}q_{\beta} $, 第四类母函数$ F_4=F_4(\vb*{p},\vb*{P},t)=F_2-\lambda\sum_{\beta=1}^s p_{\beta}q_{\beta} $, 此时有
\begin{align*}
    \dd{F_2}(\vb*{q},\vb*{P},t) &= \lambda \sum_{\beta=1}^s p_{\beta} \dd{q_{\beta}} + \sum_{\beta=1}^s Q_{\beta} \dd{P_{\beta}} + \big(\hami_{\vb*{Q}\vb*{P}}(\vb*{Q},\vb*{P},t) - \lambda \hami(\vb*{q},\vb*{p},t)\big) \dd{t},\\
    \dd{F_3}(\vb*{p},\vb*{Q},t) &= -\lambda \sum_{\beta=1}^s q_{\beta} \dd{p_{\beta}} - \sum_{\beta=1}^s P_{\beta} \dd{Q_{\beta}} + \big(\hami_{\vb*{Q}\vb*{P}}(\vb*{Q},\vb*{P},t) - \lambda \hami(\vb*{q},\vb*{p},t)\big) \dd{t},\\
    \dd{F_4}(\vb*{p},\vb*{P},t) &= - \lambda \sum_{\beta=1}^s q_{\beta} \dd{p_{\beta}} + \sum_{\beta=1}^s Q_{\beta} \dd{P_{\beta}} + \big(\hami_{\vb*{Q}\vb*{P}}(\vb*{Q},\vb*{P},t) - \lambda \hami(\vb*{q},\vb*{p},t)\big) \dd{t}.
\end{align*}
当$ \lambda=1 $时即为正则变换, $ \lambda \neq 1 $时称为\textbf{拓展正则变换}. 我们将含标度因子的正则变换母函数的性质列在表 \ref{含标度因子的正则变换母函数的性质} 中. 如果母函数$ F_i $不显含时间, 则$ \hami_{\vb*{Q}\vb*{P}}=\lambda\hami $. 一些常见的正则变换: 恒等变换$ F_2(\vb*{q},\vb*{P},t)=\sum_{\beta=1}^s q_{\beta}P_{\beta} $或$ F_3(\vb*{p},\vb*{Q},t)=-\sum_{\beta=1}^s p_{\beta}Q_{\beta} $; 对偶变换$ F_1(\vb*{q},\vb*{Q},t)=\pm\sum_{\beta=1}^s q_{\beta}Q_{\beta} $或$ F_4(\vb*{p},\vb*{P},t)=\pm\sum_{\beta=1}^s p_{\beta}P_{\beta} $; 平移变换$ F_2(\vb*{q},\vb*{P},t)=\sum_{\beta=1}^s (q_{\beta}P_{\beta}+a_{\beta}P_{\beta}-b_{\beta}q_{\beta}) $或$ F_3(\vb*{p},\vb*{Q},t)=\sum_{\beta=1}^s (-p_{\beta}Q_{\beta}+a_{\beta}p_{\beta}-b_{\beta}Q_{\beta}) $; 点变换$ F_2(\vb*{q},\vb*{P},t)=\sum_{\beta=1}^s Q_{\beta}(\vb*{q},t) P_{\beta} $.
\begin{longtable}[c]{cccc}
    \caption{含标度因子的正则变换母函数的性质 \label{含标度因子的正则变换母函数的性质}}\\
    \toprule $ F_1=F_1(\vb*{q},\vb*{Q},t) $ & $ F_2=F_2(\vb*{q},\vb*{P},t) $ & $ F_3=F_3(\vb*{p},\vb*{Q},t) $ & $ F_4=F_4(\vb*{p},\vb*{P},t) $ \\
    \midrule\endfirsthead\multicolumn{4}{r}{\small 表 \ref{含标度因子的正则变换母函数的性质} (续)} \\
    \toprule $ F_1=F_1(\vb*{q},\vb*{Q},t) $ & $ F_2=F_2(\vb*{q},\vb*{P},t) $ & $ F_3=F_3(\vb*{p},\vb*{Q},t) $ & $ F_4=F_4(\vb*{p},\vb*{P},t) $ \\ \\
    \midrule\endhead \bottomrule \endfoot \bottomrule \endlastfoot
    $ p_{\alpha}=\lambda\pdv*{F_1}{q_{\alpha}} $ & $ p_{\alpha}=\lambda\pdv*{F_2}{q_{\alpha}} $ & $ q_{\alpha}=-\lambda\pdv*{F_3}{p_{\alpha}} $ & $ q_{\alpha}=-\lambda\pdv*{F_4}{p_{\alpha}} $ \\
    $ P_{\alpha}=-\pdv*{F_1}{Q_{\alpha}} $ & $ Q_{\alpha}=\pdv*{F_2}{P_{\alpha}} $ & $ P_{\alpha}=-\pdv*{F_3}{Q_{\alpha}} $ & $ Q_{\alpha}=\pdv*{F_4}{P_{\alpha}} $ \\
    $ \hami_{\vb*{Q}\vb*{P}}=\lambda\hami+\pdv*{F_1}{t} $ & $ \hami_{\vb*{Q}\vb*{P}}=\lambda\hami+\pdv*{F_2}{t} $ & $ \hami_{\vb*{Q}\vb*{P}}=\lambda\hami+\pdv*{F_3}{t} $ & $ \hami_{\vb*{Q}\vb*{P}}=\lambda\hami+\pdv*{F_4}{t} $
\end{longtable}

\begin{proposition}[辛标记]\label{pro:辛标记}
    我们引入辛标记表示正则变换. 先回顾向量的偏导数规则: 若$ \vb*{x}=\transpose{(x_1,x_2,\ldots,x_s)} $, $ \vb*{y}=\transpose{(y_1,y_2,\ldots,y_t)} $, 则$ \pdv*{y}{\vb*{x}}=\transpose{(\pdv*{y}{x_1},\pdv*{y}{x_2},\ldots,\pdv*{y}{x_s})} $, $ \pdv*{\vb*{y}}{x}=\transpose{(\pdv*{y_1}{x},\pdv*{y_2}{x},\ldots,\pdv*{y_s}{x})} $, $ \pdv*{\vb*{y}}{\vb*{x}} $为雅可比矩阵$ \pdv*{(y_1,y_2,\ldots,y_t)}{(x_1,x_2,\ldots,x_s)} $. 考虑哈密顿函数$ \hami(\vb*{q},\vb*{p},t) $和$ \hami_{\vb*{QP}}(\vb*{Q},\vb*{P},t) $, 记列向量$ \vb*{\xi}=\transpose{(q_1,q_2,\ldots,q_s,p_1,p_2,\ldots,p_s)} $, 列向量$ \vb*{\varXi}=\transpose{(Q_1,Q_2,\ldots,Q_s,P_1,P_2,\ldots,P_s)} $, 将一对正则变换记作$ \vb*{\varXi}=\vb*{\varXi}(\vb*{\xi},t) $和$ \vb*{\xi}=\vb*{\xi}(\vb*{\varXi},t) $, 雅可比矩阵$ \pdv*{\vb*{\varXi}}{\vb*{\xi}} $记作$ \vb*{M} $, 雅可比矩阵$ \pdv*{\vb*{\xi}}{\vb*{\varXi}} $记作$ \vb*{W} $ (即$ \vb*{M}^{-1} $). 对第一类母函数$ F_1=F_1(\vb*{q},\vb*{Q},t) $作全微分, 由于$ \vb*{Q}=\vb*{Q}(\vb*{q},\vb*{p},t) $, 故考虑函数$ \tilde{F_1}=\tilde{F_1}(\vb*{q},\vb*{p},t)=F_1(\vb*{q},\vb*{Q}(\vb*{q},\vb*{p},t),t) $的全微分. 任取$ \alpha,\beta \in \{1,2,\ldots,s\} $, 求算$ \pdv{\tilde{F_1}}{\vb*{q}_{\alpha}},\pdv{\tilde{F_1}}{\vb*{q}_{\beta}},\pdv{\tilde{F_1}}{\vb*{p}_{\alpha}},\pdv{\tilde{F_1}}{\vb*{p}_{\beta}} $, 又由$ \pdv{\tilde{F_1}}{\vb*{q}_{\alpha}}{\vb*{q}_{\beta}}=\pdv{\tilde{F_1}}{\vb*{q}_{\beta}}{\vb*{q}_{\alpha}} $和$ \pdv{\tilde{F_1}}{\vb*{q}_{\alpha}}{\vb*{p}_{\beta}}=\pdv{\tilde{F_1}}{\vb*{p}_{\beta}}{\vb*{q}_{\alpha}} $和$ \pdv{\tilde{F_1}}{\vb*{p}_{\alpha}}{\vb*{p}_{\beta}}=\pdv{\tilde{F_1}}{\vb*{p}_{\beta}}{\vb*{p}_{\alpha}} $, 得方程
    \begin{equation*}
        \sum_{\gamma=1}^s \pdv{P_{\gamma}}{q_{\alpha}}\pdv{Q_{\gamma}}{q_{\beta}} = \sum_{\gamma=1}^s \pdv{P_{\gamma}}{q_{\beta}}\pdv{Q_{\gamma}}{q_{\alpha}}, \quad
        \sum_{\gamma=1}^s \pdv{P_{\gamma}}{q_{\alpha}}\pdv{Q_{\gamma}}{p_{\beta}} = \sum_{\gamma=1}^s \pdv{P_{\gamma}}{p_{\beta}}\pdv{Q_{\gamma}}{q_{\alpha}} - \kron{\alpha}{\beta}, \quad
        \sum_{\gamma=1}^s \pdv{P_{\gamma}}{p_{\alpha}}\pdv{Q_{\gamma}}{p_{\beta}} = \sum_{\gamma=1}^s \pdv{P_{\gamma}}{p_{\beta}}\pdv{Q_{\gamma}}{p_{\alpha}}.
    \end{equation*}
    将三个方程分别表示为矩阵方程后重新整合, 得到正则变换的辛条件
    \begin{equation}
        \transpose{\vb*{M}}\vb*{\varOmega}\vb*{M} = \mqty(\transpose{\big(\pdv{\vb*{Q}}{\vb*{q}}\big)} & \transpose{\big(\pdv{\vb*{P}}{\vb*{q}}\big)} \\ \transpose{\big(\pdv{\vb*{Q}}{\vb*{p}}\big)} & \transpose{\big(\pdv{\vb*{P}}{\vb*{p}}\big)}) \mqty(\vb*{0} & \vb*{I}_s\\ -\vb*{I}_s & \vb*{0}) \mqty(\pdv{\vb*{Q}}{\vb*{q}} & \pdv{\vb*{Q}}{\vb*{p}} \\ \pdv{\vb*{P}}{\vb*{q}} & \pdv{\vb*{P}}{\vb*{p}}) = \mqty(\vb*{0} & \vb*{I}_s\\ -\vb*{I}_s & \vb*{0}) = \vb*{\varOmega}.
    \end{equation}
    注意, $ \vb*{\varOmega} $是可逆的反对称矩阵, 有$ \vb*{\varOmega}^{-1}=\transpose{\vb*{\varOmega}}=-\vb*{\varOmega} $, $ \vb*{\varOmega}^2=-\vb*{I}_{2s} $. 雅可比矩阵$ \vb*{M} $称为辛矩阵, 所有正则变换的雅可比矩阵构成的集合在矩阵乘法下构成辛群$ \mathop{\mathrm{Sp}}(2s,\mathbb{R}) $. 如果正则变换$ f \colon \vb*{\xi} \mapsto \vb*{\varXi} $的辛矩阵为$ \vb*{M}_1 $, 正则变换$ g \colon \vb*{\varXi} \mapsto \vb*{\varXi}' $的辛矩阵为$ \vb*{M}_2 $, 则正则变换$ g \circ f \colon \vb*{\xi} \mapsto \vb*{\varXi}' $的辛矩阵为$ \vb*{M}_2 \vb*{M}_1 $. 另外, 由普法夫值$ \pfaff(\transpose{\vb*{M}}\vb*{\varOmega}\vb*{M})=\det(\vb*{M})\pfaff(\vb*{\varOmega})=\pfaff(\vb*{\varOmega}) = \pm 1 $, 得$ \det(\vb*{M})=1 $, 即辛矩阵的行列式值为1. 事实上, 辛标记也可用来表示哈密顿方程, 即$ \dot{\vb*{\xi}}=\vb*{\varOmega}(\pdv*{\hami}{\vb*{\xi}}) $. 更多辛标记的应用可以参考相关数学物理专著.
\end{proposition}

如果能够通过正则变换使新哈密顿函数恒等于0, 那么新哈密顿函数的变量都变为常数, 这为确定力学体系的运动方程提供了另一种手段. 考虑从哈密顿函数$ \hami(\vb*{q},\vb*{p},t) $到新哈密顿函数$ \hami_*(\vb*{Q},\vb*{P},t) $的正则变换母函数$ S $, 由于四类母函数都具有$ \hami_*(\vb*{Q},\vb*{P},t) = \hami(\vb*{q},\vb*{p},t) + \pdv*{S}{t} $的形式, 要使$ \hami_*(\vb*{Q},\vb*{P},t)=0 $, 则$ S $需满足$ \hami(\vb*{q},\vb*{p},t) + \pdv*{S}{t} = 0 $. 由于$ \vb*{Q} $和$ \vb*{P} $均为常数, 分别记作$ \vb*{C} $和$ \vb*{D} $. 考虑第二类母函数$ S=S(\vb*{q},\vb*{P},t) $, 由$ p_{\alpha}=\pdv*{S}{q_{\alpha}} $和$ Q_{\alpha}=\pdv*{S}{D_{\alpha}}=C_{\alpha} $, 得到\textbf{哈密顿}--\textbf{雅可比方程}
\begin{equation}
    \hami\bigg(\vb*{q},\pdv{S}{\vb*{q}},t\bigg) + \pdv{S(\vb*{q},\vb*{D},t)}{t} = 0.
\end{equation}
哈密顿{--}雅可比方程是函数$ S(\vb*{q},\vb*{D},t) $的一阶偏微分方程, 方程共有$ s+1 $个自变量(即$ \vb*{q} $和$ t $), 因此也会有$ s+1 $个独立的积分常数. 由于函数$ S $仅以其导数的形式出现在方程中, 故积分常数中会有$ 1 $个以相加方式出现的, 不妨考虑通积分$ S=S(\vb*{q},\vb*{D},t)+T $, 忽略$ T $不会对变换产生影响, 因而只需考虑剩下$ s $个自变量$ \vb*{D} $. 考察$ \dv*{S}{t} = \sum_{\beta=1}^s (\pdv*{S}{q_{\beta}})\dot{q_{\beta}} + \pdv*{S}{t} = \sum_{\beta=1}^s p_{\beta}\dot{q_{\beta}} - \hami(\vb*{q},\pdv*{S}{\vb*{q}},t) = \lagr $, 知$ S(\vb*{q},\vb*{D},t) = \int \lagr \dd{t} $为积分限不定的哈密顿作用量. 当哈密顿函数不显含时间时, 由体系广义能量为定值$ E $, 得$ \hami(\vb*{q},\pdv*{S}{\vb*{q}}) + \pdv*{S}{t} = E + \pdv*{S}{t} = 0 $, 哈密顿作用量可分离变量满足$ S=S(\vb*{q},\vb*{D},t)=W(\vb*{q},\vb*{D})+T(t)=W(\vb*{q},\vb*{D})-Et $, 得到不含时哈密顿{--}雅可比方程
\begin{equation}
    \hami\bigg(\vb*{q},\pdv{W}{\vb*{q}}\bigg) = E.
\end{equation}
考虑到$ \hami(\vb*{q},\pdv*{S}{\vb*{q}}) = E $, 有$ S = \int(\sum_{\beta=1}^s p_{\beta} \dot{q_\beta} - \hami)\dd{t} = \int\sum_{\beta=1}^s p_{\beta} \dd{q_{\beta}} -Et + S_0 $, 其中$ S_0 $为可略去的积分常数. 因此, $ W(\vb*{q},\vb*{D}) = \int \sum_{\beta=1}^s p_{\beta} \dd{q_{\beta}} $为积分限不定的拉格朗日作用量.

\begin{proposition}[泊松括号]\label{pro:泊松括号}
    令$ e=e(\vb*{q},t) $, $ f=f(\vb*{q},\vb*{p},t) $, $ g=g(\vb*{q},\vb*{p},t) $, $ h=h(\vb*{q},\vb*{p},t) $, $ \vb*{x}=\transpose{(q_1,q_2,\ldots,q_s,p_1,p_2,\ldots,p_s)} $, $ C_1,C_2 \in \mathbb{R} $. 我们称
    \begin{equation*}
        \pb{f}{g} = \sum_{\alpha=1}^s \bigg(\pdv{f}{q_{\alpha}}\pdv{g}{p_{\alpha}} - \pdv{g}{q_{\alpha}}\pdv{f}{p_{\alpha}}\bigg)=\sum_{\alpha=1}^s \mdet{\pdv{f}{q_{\alpha}} & \pdv{f}{p_{\alpha}} \\ \pdv{g}{q_{\alpha}} & \pdv{g}{p_{\alpha}}} = \sum_{\alpha=1}^s \det[\pdv{(f,g)}{(q_{\alpha},p_{\alpha})}] = \transpose{\bigg(\pdv{f}{\vb*{x}}\bigg)} \vb*{\varOmega} \pdv{g}{\vb*{x}}
    \end{equation*}
    为泊松括号, 称$ \pb{q_{\alpha}}{q_{\beta}}=\pb{p_{\alpha}}{p_{\beta}}=0 $和$ \pb{q_{\alpha}}{p_{\beta}}=\kron{\alpha}{\beta} $为基本泊松记号, 其中$ \kron{\alpha}{\beta} $为克罗内克记号.
    \begin{enumerate}
        \item 对任意$ \alpha \in \{1,2,\ldots,s\} $, 有$ \pdv*{\dot{e}}{\dot{q_{\alpha}}} = \pdv*{e}{q_{\alpha}} $和$ \dv{t}(\pdv*{e}{q_{\alpha}}) = \pdv*{\dot{e}}{q_{\alpha}} $.
        \item 对任意$ \alpha \in \{1,2,\ldots,s\} $, 有$ \pb{q_{\alpha}}{f}=\pdv*{f}{p_{\alpha}} $和$ \pb{p_{\alpha}}{f}=-\pdv*{f}{q_{\alpha}} $.
        \item $ \pb{f}{g}=-\pb{g}{f} $, $ \pb{f}{C_1g+C_2h}=C_1\pb{f}{g}+C_2\pb{f}{g} $, $ \pb{f}{gh}=\pb{f}{g}h+g\pb{f}{h} $.
        \item 对任意$ x \in \{q_1,q_2,\ldots,q_s,p_1,p_2,\ldots,p_s,t\} $, 有$ \pdv*{\pb{f}{g}}{x} = \pb{\pdv*{f}{x}}{g} + \pb{f}{\pdv*{g}{x}} $.
        \item 雅可比恒等式$ \pb{f}{\pb{g}{h}}+\pb{g}{\pb{h}{f}}+\pb{h}{\pb{f}{g}}=0 $.
    \end{enumerate}
\end{proposition}

泊松括号在研究守恒量相关问题上有较大用处. 考虑函数$ f=f(\vb*{q},\vb*{p},t) $, 有$ \dv*{f}{t} = \pdv*{f}{t}+\sum_{\beta=1}^s \big(\pdv{f}{q_{\beta}}\dot{q_{\beta}}+\pdv{f}{p_{\beta}}\dot{p_{\beta}}\big) = \pdv*{f}{t}+\pb{f}{\hami(\vb*{q},\vb*{p},t)} $. 如果函数$ f $不显含时间, 则有$ \dv*{f}{t} = \pb{f}{\hami(\vb*{q},\vb*{p},t)} $. 当力学体系状态随时间演变时保持不变的力学量$ f $称为\textbf{运动积分}. 显然, 运动积分$ f $满足$ \dv*{f}{t}=\pb{f}{\hami(\vb*{q},\vb*{p},t)}=0 $. \textbf{泊松定理}告诉我们: 如果函数$ f,g $是运动积分, 则$ \pb{f}{g} $也是运动积分. 由于运动积分$ f,g $有$ \pdv*{f}{t}=-\pb{f}{\hami} $和$ \pdv*{g}{t}=-\pb{g}{\hami} $, 得$ \dv{t} \pb{f}{g} = \pdv{t} \pb{f}{g}+\pb{\pb{f}{g}}{\hami} = \pb{\pdv*{f}{t}}{g}+\pb{f}{\pdv*{g}{t}}+\pb{\pb{f}{g}}{\hami} = \pb{-\pb{f}{\hami}}{g}+\pb{f}{-\pb{g}{\hami}}+\pb{\pb{f}{g}}{\hami} = -\pb{f}{\pb{g}{\hami}}-\pb{g}{\pb{\hami}{f}}-\pb{\hami}{\pb{f}{g}} = 0 $, 因此$ \pb{f}{g} $也是运动积分. 泊松括号在正则变换下是保持不变的, 考虑函数$ f=f_{\xi}(\vb*{\xi},t)=f_{\varXi}(\vb*{\varXi},t) $和$ g=g_{\xi}(\vb*{\xi},t)=g_{\varXi}(\vb*{\varXi},t) $, 有
\begin{equation*}
    \pb{f}{g}_{\varXi} = \transpose{\bigg(\pdv{f}{\vb*{\varXi}}\bigg)} \vb*{\varOmega} \pdv{g}{\vb*{\varXi}} = \transpose{\bigg(\pdv{f}{\vb*{\xi}}\pdv{\vb*{\xi}}{\vb*{\varXi}}\bigg)} \vb*{\varOmega} \bigg(\pdv{g}{\vb*{\xi}}\pdv{\vb*{\xi}}{\vb*{\varXi}}\bigg) = \transpose{\bigg(\pdv{f}{\vb*{\xi}}\bigg)} \transpose{(\vb*{M}^{-1})} \vb*{\varOmega}\vb*{M}^{-1} \pdv{g}{\vb*{\xi}} = \transpose{\bigg(\pdv{f}{\vb*{\xi}}\bigg)} \vb*{\varOmega} \pdv{g}{\vb*{\xi}} = \pb{f}{g}_{\xi}.
\end{equation*}
相应地, 基本泊松括号在正则变化下也是保持不变的, 这可以作为正则变换的判据.

我们可以考虑由力学体系的位形或状态张成的空间的性质. 由广义坐标$ \vb*{q} $张成的$ s $维空间$ C $称为\textbf{位形空间}. 力学体系的每个可能的位形都对应于$ C $中一个点, 体系的位形随时间的演化相当于在$ C $中画出一条连续的轨迹. 由广义坐标$ \vb*{q} $和广义速度$ \dot{\vb*{q}} $张成的$ 2s $维空间$ S $称为\textbf{状态空间}. 力学体系的每个可能的状态都对应于$ S $中一个点, 体系的状态随时间的演化相当于在$ S $中画出一条连续的轨迹. 由广义坐标$ \vb*{q} $和广义动量$ \vb*{p} $张成的$ 2s $维空间$ \varGamma $称为\textbf{相空间}. 力学体系的每个可能的状态都对应于$ \varGamma $中一个点(称为\textbf{相点}), 体系的状态随时间的演化相当于在$ \varGamma $中画出一条连续的轨迹(称为\textbf{相轨道}). 相体积在正则变换下是保持不变的, 考虑相体积元$ \prod_{\beta=1}^{2s} \dd{\xi_{\beta}}=\dd{q_1}\dd{q_2}\cdots\dd{q_s}\dd{p_1}\dd{p_2}\cdots\dd{p_s} $和$ \prod_{\beta=1}^{2s} \dd{\varXi_{\beta}}=\dd{Q_1}\dd{Q_2}\cdots\dd{Q_s}\dd{P_1}\dd{P_2}\cdots\dd{P_s} $, 有\textbf{刘维尔定理}
\begin{equation*}
    \int_{\varLambda^*} \prod_{\beta=1}^{2s} \dd{\varXi_{\beta}} = \int_{\varLambda} \abs{\det\bigg(\pdv{\vb*{\varXi}}{\vb*{\xi}}\bigg)} \prod_{\beta=1}^{2s} \dd{\xi_{\beta}} = \int_{\varLambda} \abs{\det(\vb*{M})} \prod_{\beta=1}^{2s} \dd{\xi_{\beta}} = \int_{\varLambda} \prod_{\beta=1}^{2s} \dd{\xi_{\beta}}.
\end{equation*}



%%%%%%%%%%%%%%%%%%%%%%%%%%%%%%%%%%%%%%%%%%%%%%%%%%%%%%%%%%%%%%%%
\section[经典场论]{经典场论}\label{经典场论}




%%%%%%%%%%%%%%%%%%%%%%%%%%%%%%%%%%%%%%%%%%%%%%%%%%%%%%%%%%%%%%%%
\newpage
\section[经典统计力学]{经典统计力学}\label{经典统计力学}
经典统计力学基于概率原理描述具有大自由度的\textbf{热力学体系}, 体系中的所有粒子遵循经典力学规律, 大自由度又使体系具有\textbf{统计规律性}. 如果体系各部分的宏观性质不随时间改变, 且不存在外部或内部的某些作用使体系内以及体系与环境之间有任何宏观流(物质流与能流)和化学反应发生, 则称体系处于\textbf{平衡态}, 否则称体系处于\textbf{非平衡态}. 如果体系各部分的宏观性质不随时间改变, 但存在宏观流(物质流或能流), 则称为\textbf{定态}. 换言之, 如果体系的\textbf{宏观状态}不随时间改变, 则称体系处于平衡态, 否则称体系处于非平衡态. 体系从非平衡态到达平衡态所需要的时间称为\textbf{弛豫时间}. 热力学体系以外的部分称为\textbf{环境}. 如果体系与环境之间不能交换物质与能量, 则称为\textbf{孤立体系}. 如果通过体系与环境的边界可交换能量, 但不能交换物质, 则称为\textbf{封闭体系}. 如果通过体系与环境的边界可交换物质与能量, 则称为\textbf{开放体系}. 热力学体系中物理性质均匀的一个宏观部分称为一个\textbf{相}, 各部分是完全均匀的体系称为\textbf{单相系}或\textbf{均相系}, 否则称为\textbf{复相系}或\textbf{非均相系}. 仅含一种化学组分的体系称为\textbf{单组分体系}, 否则称为\textbf{多组分体系}. 体系中粒子的种类和数目始终保持不变的体系称为\textbf{组成恒定体系}. 如果体系的宏观状态随时间发生改变, 则称体系经历了从\textbf{始态}到\textbf{终态}的\textbf{热力学过程}. 如果体系经历一个无限缓慢的热力学过程, 在过程的任意时刻体系都处于平衡态, 则称该过程\textbf{准静态过程}. 如果体系经历的热力学过程对体系和环境产生的影响能够在不引起其他变化的情况下完全消除, 则称该过程为\textbf{可逆过程}, 否则称为\textbf{不可逆过程}. 无耗散的准静态过程是可逆过程.

体系的宏观状态可由\textbf{宏观参量}描述, 处于平衡态的体系的宏观参量具有确定的数值. 常见的宏观参量包括几何参量(如长度, 面积, 体积, 形变等), 力学参量(如力, 压强, 胁强等), 电磁参量(如电场强度, 电极化强度, 磁场强度, 磁化强度等), 热学参量(如温度, 热力学能, 熵, 焓等), 化学参量(如组分浓度, 化学势等). 处于平衡态的均相系的独立宏观参量的数目$ F $满足$ F=R+\omega+1 $, 其中$ R $是体系的可变物种数, $ \omega $是可逆功的形式数, $ 1 $来源于热交换. 通常将描述体系宏观状态的独立宏观参量称为体系的\textbf{状态变量}, 以状态变量为参数的函数称为体系的\textbf{状态函数}. 体系状态函数的改变量只取决于平衡始态和平衡终态, 与体系经历的热力学过程无关. 描述处于平衡态的均相系的各宏观参量之间关系的方程称为该体系的\textbf{物态方程}.

体系的微观状态分为纯态和混合态两类. 如果体系的微观状态可以由相空间$ \varGamma $的广义坐标$ \vb*{q} $和广义动量$ \vb*{p} $确定地描述, 则称微观状态处于\textbf{纯态}. 如果体系的微观状态由具有指定概率分布的一组微观状态概率性地描述, 则称微观状态处于\textbf{混合态}, 其中的微观状态的概率分布由概率密度函数指定. 从相空间的角度考虑, 处于纯态的微观状态就是一个相点, 而处于混合态的微观状态是一团相概率云. 令$ \dd{N}(\vb*{q},\vb*{p},t) $表示$ t $时刻相点$ (\vb*{q},\vb*{p}) $周围的相体积元$ \dd{\varGamma} $内相点的数量, 相空间的\textbf{概率密度函数}为
\begin{equation}
    \rho(\vb*{q},\vb*{p},t) = \lim_{N \to \infty} \frac{1}{N}\dv{N(\vb*{q},\vb*{p},t)}{\varGamma},
\end{equation}
显然概率密度函数满足$ \int_{\varGamma} \rho(\vb*{q},\vb*{p},t) \dd{\varGamma} =1 $.

我们现在考虑如何构建体系的微观状态和宏观状态之间的联系. 当我们测量体系的宏观参量时, 测量空间尺度是宏观小的(宏观参量具有确定值)而微观大的(粒子的数量足够多), 测量时间尺度是宏观短的(宏观参量具有确定值)而微观长的(粒子的微观状态已发生变化). 相空间$ \varGamma $中的相点在测量过程中必然会发生移动, 我们对宏观参量的测量值实际上是测量时间范围$ t_0 \leq t \leq t_0 + \tau $内众多微观量$ B(\vb*{q}(t),\vb*{p}(t)) $的\textbf{时间平均值}$ \langle B(t_0) \rangle_{\mathrm{time}} = \frac{1}{\tau} \int_{t_0}^{t_0+\tau} B(\vb*{q}(t),\vb*{p}(t))\dd{t} $. 但由于我们无法得到微观状态的实际相轨道表达式, 时间平均值也就不能计算. 我们转而考虑大量处于相同的宏观条件下, 具有相同宏观状态的相互独立的体系构成的集合, 即\textbf{系综}. 系综在相空间里表现为大量相点构成的区域, 这些相点具有相同的宏观状态. 具有给定宏观状态的体系的微观状态可以视作处于指定概率分布的一组系综内体系的微观状态, 此时该体系的微观状态处于混合态. 我们期望系综内所有体系在给定时刻下的微观状态包含相同宏观状态的体系在不同时刻的所有微观状态, 此时可以将体系在不同时刻的微观量的时间平均转化为系综内体系在同一时刻的微观量的系综平均, 这就解决了微观状态随时间演化对求宏观参量的困难. 考虑系综内体系在$ t_0 $时刻的微观量$ B(\vb*{q}(t_0),\vb*{p}(t_0)) $的\textbf{系综平均值}$ \langle B(t_0) \rangle_{\mathrm{ens}} = \int_{\varGamma} \rho(\vb*{q},\vb*{p},t_0) B(\vb*{q}(t_0),\vb*{p}(t_0)) \dd{\varGamma} $为系综的\textbf{宏观参量}, 我们引入\textbf{遍历性假设}后, 体系的宏观参量即为系综的宏观参量. 通常遍历性假设仅对封闭体系和保守体系有效, 但即使不引入遍历性假设, 我们也可以发展一套关于系综宏观性质的统计物理.

\begin{postulate}[遍历性假设]\label{pos:遍历性假设}
    微观量的时间平均值与系综平均值相等, 即
    \begin{equation}
        \langle B(t_0) \rangle_{\mathrm{time}} = \frac{1}{\tau} \int_{t_0}^{t_0+\tau} B(\vb*{q}(t),\vb*{p}(t))\dd{t} = \int_{\varGamma} \rho(\vb*{q},\vb*{p},t_0) B(\vb*{q}(t_0),\vb*{p}(t_0)) \dd{\varGamma} = \langle B(t_0) \rangle_{\mathrm{ens}},
    \end{equation}
    记$ \avg{B(t_0)} = \langle B(t_0) \rangle_{\mathrm{time}} = \langle B(t_0) \rangle_{\mathrm{ens}} $.
\end{postulate}

我们可以考虑相空间中概率密度函数$ \rho(\vb*{q},\vb*{p},t) $随时间的演化规律, 此时需要额外假设相空间内的总相点数$ N $是守恒的, 不存在相点的净产生或净消失, 即$ \dv*{N}{t}=0 $. 令体积元$ \dd{\varGamma} = \dd{q_1} \cdots \dd{q_s} \dd{p_1} \cdots \dd{p_s} $, $ \alpha \in \{1,2,\ldots,s\} $, 广义坐标平面$ q_{\alpha} $的面积元$ \dd{A_{\alpha}} = \dd{q_1} \cdots \dd{q_{\alpha-1}} \dd{q_{\alpha+1}} \cdots \dd{q_s} \cdots \dd{p_1} \cdots \dd{p_s} $, 广义动量平面$ p_{\alpha} $的面积元$ \dd{B_{\alpha}} = \dd{q_1} \cdots \dd{q_s} \cdots \dd{p_1} \cdots \dd{p_{\alpha-1}} \dd{p_{\alpha+1}} \cdots \dd{p_s} $. 在$ \dd{t} $内通过平面$ q_{\alpha} $进入$ \dd{\varGamma} $的相点数为$ N \rho \dot{q_{\alpha}} \dd{A_{\alpha}} \dd{t} $, 在$ \dd{t} $内通过平面$ q_{\alpha} $离开$ \dd{\varGamma} $的相点数为$ N (\rho \dot{q_{\alpha}} + \pdv{q_{\alpha}} (\rho \dot{q_{\alpha}}) \dd{q_{\alpha}}) \dd{A_{\alpha}} \dd{t} $, 因此在$ \dd{t} $内通过平面$ q_{\alpha} $造成的$ \dd{\varGamma} $内相点数净变化量为$ - N \pdv{q_{\alpha}} (\rho \dot{q_{\alpha}}) \dd{q_{\alpha}} \dd{A_{\alpha}} \dd{t} $. 类似地, 在$ \dd{t} $内通过$ p_{\alpha} $造成的$ \dd{\varGamma} $内相点数净变化量为$ - N \pdv{p_{\alpha}} (\rho \dot{p_{\alpha}}) \dd{p_{\alpha}} \dd{B_{\beta}} \dd{t} $. 现在考虑所有的广义坐标平面和广义动量平面, 得到在$ \dd{t} $内体积元$ \dd{\varGamma} $内相点数净变化总量为$ - \sum_{\beta=1}^s N (\pdv{q_{\beta}} (\rho \dot{q_{\beta}}) \dd{q_{\beta}} \dd{A_{\beta}} + \pdv{p_{\beta}} (\rho \dot{p_{\beta}}) \dd{p_{\beta}} \dd{B_{\beta}}) \dd{t} = -\sum_{\beta=1}^s N (\pdv{q_{\beta}} (\rho \dot{q_{\beta}}) + \pdv{p_{\beta}} (\rho \dot{p_{\beta}}) ) \dd{\varGamma}\dd{t} $. 如果令$ \vb*{v}=(\dot{q_1},\ldots,\dot{q_s},\dot{p_1},\ldots,\dot{p_s}) $, 则$ -\sum_{\beta=1}^s N (\pdv{q_{\beta}} (\rho \dot{q_{\beta}}) + \pdv{p_{\beta}} (\rho \dot{p_{\beta}}) ) \dd{\varGamma}\dd{t} = -N \div{(\rho \vb*{v})} \dd{\varGamma}\dd{t} = -N (\grad{\rho} \cdot \vb*{v} + \rho \div{\vb*{v}}) \dd{\varGamma}\dd{t} $. 又由于在$ \dd{t} $内体积元$ \dd{\varGamma} $内相点数净变化总量为$ N (\pdv*{\rho}{t}) \dd{\varGamma}\dd{t} $, 得
\begin{align*}
    -\pdv{\rho}{t} &= \grad{\rho} \cdot \vb*{v} + \rho \div{\vb*{v}} = \sum_{\beta=1}^s \bigg(\pdv{\rho}{q_{\beta}} \dot{q_{\beta}} + \pdv{\rho}{p_{\beta}} \dot{p_{\beta}} + \pdv{\dot{q_{\beta}}}{q_{\beta}} \rho + \pdv{\dot{p_{\beta}}}{p_{\beta}} \rho \bigg)\\
    &= \sum_{\beta=1}^s \bigg(\pdv{\rho}{q_{\beta}} \dot{q_{\beta}} + \pdv{\rho}{p_{\beta}} \dot{p_{\beta}} + \pdv{\hami}{q_{\beta}}{p_{\beta}} \rho - \pdv{\hami}{p_{\beta}}{q_{\beta}} \rho \bigg) = \sum_{\beta=1}^s \bigg(\pdv{\rho}{q_{\beta}} \dot{q_{\beta}} + \pdv{\rho}{p_{\beta}} \dot{p_{\beta}} \bigg) = \pb{\rho}{\hami}.
\end{align*}
因此, $ \dv*{\rho}{t} = \pdv*{\rho}{t}+ \div{(\rho \vb*{v})} = \pdv*{\rho}{t}+\pb{\rho}{\hami} = 0 $, 即\textbf{刘维尔方程}. 相空间的概率密度的行为类似不可压缩流体, 刘维尔方程也具有流体连续性方程的形式. 如果选取相空间中的相点$ (\vb*{q},\vb*{p}) $, 经过$ \dd{t} $后该相点抵达新的相点$ (\vb*{q}+\dd{\vb*{q}},\vb*{p}+\dd{\vb*{p}}) $并形成了一条相轨道, 现在我们沿该相轨道跟随相点移动, 并观察所处相点周围相体积元$ \dd{\varGamma} $内相点的数目随所处时间的变化, 可以发现所处相点周围体积元内相点数目总是恒定的, 即$ \rho(\vb*{q},\vb*{p},t)=\rho(\vb*{q}+\dd{\vb*{q}},\vb*{p}+\dd{\vb*{p}},t+\dd{t}) $. 我们现在可以考虑系综平均值$ \avg{B(t)} $随时间的演化规律
\begin{align*}
    \dv{\avg{B(t)}}{t} &= \int_{\varGamma} \pdv{\rho(\vb*{q},\vb*{p},t)}{t} B(\vb*{q},\vb*{p}) \dd{\varGamma} = \int_{\varGamma} \sum_{\beta=1}^s \bigg(\pdv{\rho}{p_{\beta}} \pdv{\hami}{q_{\beta}} - \pdv{\rho}{q_{\beta}} \pdv{\hami}{p_{\beta}} \bigg) B(\vb*{q},\vb*{p}) \dd{\varGamma}\\
    &=-\int_{\varGamma} \rho(\vb*{q},\vb*{p},t) \sum_{\beta=1}^s \bigg(\pdv{B}{p_{\beta}} \pdv{\hami}{q_{\beta}} - \pdv{B}{q_{\beta}} \pdv{\hami}{p_{\beta}} + B(\vb*{q},\vb*{p})\bigg( \pdv{\hami}{p_{\beta}}{q_{\beta}} - \pdv{\hami}{q_{\beta}}{p_{\beta}} \bigg) \bigg) \dd{\varGamma}\\
    &=-\int_{\varGamma} \rho(\vb*{q},\vb*{p},t) \pb{\hami}{B} \dd{\varGamma} = \avg{\pb{B}{\hami}}.
\end{align*}
特别地, 如果体系处于平衡态, 则$ \rho_{\eqv} $为运动积分, $ \pdv*{\rho_{\eqv}}{t}=\pb{\rho_{\eqv}}{\hami}=0 $, $ \rho_{\eqv}(\vb*{q},\vb*{p},t)=\rho_{\eqv}(\vb*{q},\vb*{p},t+\dd{t})=\rho_{\eqv}(\vb*{q}+\dd{\vb*{q}},\vb*{p}+\dd{\vb*{p}},t+\dd{t}) $, $ \dv*{\avg{B(t)}}{t}=\avg{\pb{B}{\hami}}=0 $.

\begin{postulate}[等几率原理]\label{pos:等几率原理}
    对处于平衡态的孤立体系, 体系各个可能的微观状态出现的几率相等.
\end{postulate}

由具有固定广义能量$ E $和广义坐标$ \vb*{x} $的相互独立的热力学体系构成的集合称为\textbf{微正则系综}, 微正则系综中体系的微观状态为$ \mu $的概率密度满足
\begin{equation}
    \rho_{(E,\vb*{x})}(\mu)=
    \begin{cases}
        1/\varOmega(E,\vb*{x}), & \text{for }\hami(\mu)=E,\\
        0, & \text{otherwise}.
    \end{cases}
\end{equation}
根据等几率原理, 微正则系综可以用来描述孤立体系的热力学性质.
